\documentclass{article}

\usepackage[cm]{fullpage}
\usepackage[russian]{babel}
\usepackage{amssymb}
\usepackage{amsmath}
\usepackage{setspace}
\usepackage{titlesec}
\usepackage{mdframed}
\usepackage{xcolor}
\usepackage{graphicx}
\usepackage{subcaption}
\usepackage{indentfirst}
\usepackage{titlesec}
\usepackage{accents}
\usepackage{hyperref}

\usepackage{float}

% Headers
\titleformat{\section}{\normalfont \Large \bfseries}{\thesection .}{0.2cm}{}
\titlespacing{\section}{0pt}{*-1}{*-1}

\titleformat{\subsection}{\normalfont \large \bfseries}{\thesubsection .}{0.2cm}{}
\titlespacing{\subsection}{3pt}{*-1}{*-2}

\titleformat{\subsubsection}{\normalfont \bfseries}{\thesubsubsection .}{0.2cm}{}
\titlespacing{\subsubsection}{6pt}{*-1}{*-2}

% Space between rows
\linespread{1.5}

% Space between paragraphs
\setlength{\parindent}{1cm}
\setlength{\parskip}{0.5cm}

% Commands
\newcommand{\R}{\mathbb R}
\newcommand{\E}[1]{\textsf E \left[ #1 \right]}
\newcommand{\Var}[1]{\textsf{Var} \left( #1 \right)}
\newcommand{\intR}{\int_{\R}}
\newcommand{\arc}[1]{\accentset{\smallsmile}{#1}}

% Operators


\begin{document}

\section{ВВЕДЕНИЕ}

\section{ТЕОРЕТИЧЕСКОЕ ОПИСАНИЕ МОДЕЛИ}

Рассмотрим модель
\[
d S_t = f \left( M_t \right) dt + \sigma_t d W_t \quad \overset{def}{\Leftrightarrow} \quad S_t = \int\limits_0^t f \big( M_s \big) ds + \int\limits_0^t \sigma_s d W_s, \quad (1)
\]
где $S_t$ - логарифм зависимой переменной, $f$ - некоторая функция от $M_t$ - матрицы ''объекты-признаки'', $\sigma_t$ - параметр, в общем случае непостоянный, $W_t$ - броуновское движение.

Заметим, что выражение $(1)$ можно представить в виде
\[
S_{k \Delta} - S_{(k - 1) \Delta} = \left( f \big( M_{k \Delta}  \big) - f \big( M_{(k - 1) \Delta} \big) \right) \Delta + \sigma_{k \Delta} \left(W_{k \Delta} - W_{(k - 1) \Delta} \right), \quad (2)
\]
где $k \Delta$ и $(k - 1) \Delta$ - некоторые моменты времени на сетке $\{\Delta, 2 \Delta, ... , n \Delta\}$. Тогда, деление обеих частей $(2)$ на $\sqrt{\Delta}$ даст
\[
\frac{ S_{k \Delta} - S_{(k - 1) \Delta}}{\sqrt{\Delta}} = \left( f \big( M_{k \Delta} \big) - f \big( M_{(k - 1) \Delta} \big) \right) \sqrt{\Delta} +  \sigma_{k \Delta} \frac{\left( W_{k \Delta} - W_{(k - 1) \Delta} \right)}{\sqrt{\Delta}}. \quad (3)
\]

Также известно, что $ W_{k \Delta} - W_{(k - 1) \Delta} \sim \mathrm{N} (0, \Delta).$ Следовательно,  $\frac{W_{k \Delta} - W_{(k - 1) \Delta}}{\sqrt{\Delta}} \sim \mathrm{N}(0, 1)$.

Поэтому
\[
\frac{ S_{k \Delta} - S_{(k - 1) \Delta}}{\sqrt{\Delta}} = \left( f \big( M_{k \Delta} \big) - f \big( M_{(k - 1) \Delta} \big) \right) \sqrt{\Delta} +  \sigma_{k \Delta} Z_{k \Delta}, \quad Z_{k \Delta} \sim \mathcal{N}(0, 1). \quad (4)
\]

\section{РЕКУРРЕНТНЫЕ НЕЙРОННЫЕ СЕТИ}

\subsection{Базовая RNN}

Для обработки последовательных данных зачастую используют рекуррентные нейронные сети (далее RNN - recurrent neural networks). Их ключевой особенностью является использование при обучении предшествующих данных наряду с текущими, что схематично представлено на Рис. 1.

\begin{figure}[H]
    \centering
    \includegraphics[width=0.5\textwidth]{/Users/mmenadzhiev/Desktop/Studying/Project/Estimation of the price/Отчет/Images/rnn.png}
    \caption{Структура простейшей RNN}
\end{figure}

В общих чертах принцип работы простейшей RNN подразумевает сохранение результатов обучения на предыдущих относительно текущего значениях и их использования для более качественного прогнозирования. Действительно, в контексте временных редов и последовательных данных в целом история очевидно играет значительную роль, поэтому такой подход является вполне разумным.

При этом базовая структура RNN может повлечь несколько значительных проблем при обучении, например, таких как \textbf{взрыв} или \textbf{затухание градиентов} (vanishing or exploding gradients). Проще всего ее осознать на конкретном примере. Так, на Рис. 2 представлен частный случай RNN, в котором за передачу информации от предыдущего значения к текущему отвечает параметр $W_2$. Также понятно, что при обратном проходе по графу (поиске производных по параметрам) в цепочке произведения частных производных мы так или иначе получим $W_2^2$. Следовательно, при числе слоев $n$ мы столкнемся с множителем $W_2^n$, и тогда если $|W_2| < 1$, то $W_2^n \to 0$ при $n \to \infty$, что называется затуханием градиента. Аналогично при $|W_2| > 1 \; W_2^n \to \infty$ при $n \to \infty$ - взрыв градиента.

\begin{figure}[H]
    \centering
    \includegraphics[width=0.5\textwidth]{/Users/mmenadzhiev/Desktop/Studying/Project/Estimation of the price/Отчет/Images/rnn2.png}
    \caption{Частный пример RNN $\text{}^1$}
\end{figure}

Заметим, что выявленные проблемы действительно значительны, так как при затухании градиента веса фактически перестают обновляться, а при взрыве - наоборот меняются слишком сильно и препятствуют нахождению оптимума, что следует из формулы
\[
w_t = w_{t - 1} - \frac 1 \eta Q(w_{t - 1}), \quad (5)
\]
где $w_t$ - вектор весов в момент $t$, $\eta$ - скорость обучения, $Q(x)$ - градиент функции потерь.

Таким образом, для решения нашей задачи необходимо более продуманная реализация RNN, не имеющая вышеописанных недостатков, например, LSTM (Long Short-Term Memory), которую мы рассмотрим в дальнейшем.

\subsection{Long Short-Term Memory}

Перейдем к описанию следующего поколения RNN - модели LSTM (Long Short-Term Memory). Данная модификация простейшей рекуррентной нейронной сети позволяет избежать такой значительной проблемы, как затухание или взрыв градиента засчет специальной структуры, схематично представленной на Рис. 3.

\begin{figure}[H]
    \centering
    \includegraphics[width=0.5\textwidth]{/Users/mmenadzhiev/Desktop/Studying/Project/Estimation of the price/Отчет/Images/LSTM.png}
    \caption{Структура LSTM}
\end{figure}

Основная идея LSTM-сетей заключается в использовании двух потоков информации, влияющих на конечный прогноз напрямую - краткосрочной и долгосрочной памяти (Рис. 4). Заметим, что в базовой RNN  долгосрочные данные имели лишь опосредованное значение через последовательное суммирование и применение функций активации (Рис. 2).

\begin{figure}[H]
    \centering
    \includegraphics[width=0.3\textwidth]{/Users/mmenadzhiev/Desktop/Studying/Project/Estimation of the price/Отчет/Images/LSTM2.png}
    \caption{Иллюстрация влияния долгосрочной и краткосрочной памяти в рамках LSTM-модели$\text{}^2$}
\end{figure}

Наконец перейдем к более подробному описанию LSTM. Для начала еще раз обратимся к Рис. 3, на котором видно, что в рамках данной модели используются две функции активации - сигмоида ($\sigma$) и гиперболический тангенс ($tanh$). Вспомним, как они задаются
\begin{enumerate}
\item Сигмоида:
\[
\sigma(x) = \frac{e^x}{1 + e^x}, \quad (6)
\]
причем $\sigma: \mathbb R \mapsto (0, 1)$.

\item Гиперболический тангенс:
\[
tanh(x) = \frac{e^x - e^{-x}}{e^x + e^{-x}}, \quad (7)
\]
причем $tanh: \mathbb R \mapsto (-1, 1)$.
\end{enumerate}

Что касается одной составляющей LSTM-модели, то ее условно можно разделить на три блока, представленных на Рис. 5, в котором розовой линией обозначена краткосрочная память, а зеленой - долгосрочная. Также функция в маленьком голубом прямоугольнике есть сигмоида, а в маленьком оранжевом - гиперболический тангенс.

Первый блок выделен большим голубым прямоугольником, в котором по существу происходит определение доли от входящего значения \textit{долгосрочной} памяти, которую необходимо оставить для дальнейшего использования.

Второй блок образован большим зеленым и большим оранжевым прямоугольниками. В большом оранжевом прямоугольнике вычисляется потенциальная \textit{долгосрочная} память, а в синем - аналогично первому блоку доля, которую модель запомнит. Затем найденное значение будет сложено с получившимся после первого блока, что образует новую \textit{долгосрочную} память, которая будем передана в дальнейшие итерации модели.

Третий блок получается из объединения большого фиолетого и большого розового прямоугольников. В нем и образуется выход, который либо передается в последующие части модели, аналогичные текущей, в качестве новой \textit{краткосрочной} памяти, либо уже представляет конечный результат. Более конкретно, в третьем блоке на основе новой \textit{долгосрочной} памяти определяется потенциальная \textit{краткосрочная}, и по известной схеме из первого блока определяется ее доля.

\begin{figure}[H]
    \centering
    \includegraphics[width=0.5\textwidth]{/Users/mmenadzhiev/Desktop/Studying/Project/Estimation of the price/Отчет/Images/LSTM3.jpg}
    \caption{Частный пример LSTM$\text{}^2$}
\end{figure}

Заметим, что для получения произвольного прогноза, например, в задаче регрессии, можно к выходу модели добавить полносвязный слой. Таким образом, LSTM является универсальной нейросетью, позволяющей работать с последовательной информацией, что актуально и для нашей задачи.

\section{ОЦЕНКА ФУНКЦИИ $f$}

\subsection{Описание компонент модели и подхода к оцениванию}

Для начала вспомним, как выглядит наша модель согласно уравнению (4)
\[
\frac{ S_{k \Delta} - S_{(k - 1) \Delta}}{\sqrt{\Delta}} = \left( f \big( M_{k \Delta} \big) - f \big( M_{(k - 1) \Delta} \big) \right) \sqrt{\Delta} +  \sigma_{k \Delta} Z_{k \Delta}, \quad Z_{k \Delta} \sim \mathcal{N}(0, 1),
\]
где $f$ - некоторая функция от $M_t$ - матрицы ''объекты-признаки'', в которой находятся некоторые временные показатели, соответствующие зависимой переменной, например, лаги порядка от 1 до $h, \, h \in \mathbb N$. Иными словами, строки матрицы $M_t$ состоят из последовательностей временных данных, для обработки которых, как уже было описано выше, возможно использование рекуррентных нейронных сетей, и в частности LSTM-модели, которую мы и будем обучать для оценки функции $f$.

Также в данном разделе предположим постоянство $\sigma$, то есть $\sigma_{\Delta k} = \sigma, \, \forall k = 0, 1, ... $. Следовательно, текущая модель выглядит как
\[
\frac{ S_{k \Delta} - S_{(k - 1) \Delta}}{\sqrt{\Delta}} = \left( f \big( M_{k \Delta} \big) - f \big( M_{(k - 1) \Delta} \big) \right) \sqrt{\Delta} +  \sigma Z_{k \Delta}, \quad Z_{k \Delta} \sim \mathcal{N}(0, 1). \quad (8)
\]

Теперь более подробно рассмотрим матрицу $M_t$. В базовом варианте она содержит исключительно лаги некоторой объясняющей переменной, коррелирующей с зависимой. Для более подробного объяснения обратимся к примеру. Пусть $y_t$ - зависимая переменная, в нашем случае $y_t = \frac{S_t - S_{t - 1}}{\sqrt{\Delta}}$, а $x_t$ - объясняющая переменная, связь которой с $y_t$ чем-либо обоснована. Тогда если $h, \, h \in \mathbb N$ - максимальный порядок лага, то матрица ''объекты-признаки'' на момент $t$ есть
\[
M_t = \begin{bmatrix}
    x_{1} & x_{2} & \cdots & x_{h} \\
    x_{2} & x_{3} & \cdots & x_{h + 1} \\
    \vdots & \vdots & \ddots & \vdots \\
    x_{t - h} & x_{t - h + 1} & \cdots & x_{t - 1} 
\end{bmatrix}_.
\]

В большинстве случаев модели машинного обучения обладают \textit{гиперпараметрами} - параметрами, не участвующими в обучении, значения которых необходимо выбрать заранее на основе каких-либо принципов, например, кросс-валидации. Что касается рассматриваемой модели, соответствующей уравнению (8), то ее \textit{гиперпараметры} есть:

\vspace{-20pt}

\begin{itemize}

\item $\sigma$ - параметр, отвечающий за волатильность

\item $\Delta$ - размерность временной сетки

\item $h$ - количество лагов объясняющей переменной, используемое при оценке

\end{itemize}

\vspace{-10pt}

Наконец определим функцию потерь, при помощи которой будет происходить оценка функции $f$. Заметим, что в таблице $M_{k - 1} \Delta$ находится лаг порядка $h$, а минимальный допустимый номер наблюдения есть 1. Следовательно, наименьший доступный $k$ равен $h + 2$. Тогда функцию потерь можно записать как
\[
L = \sum\limits_{k = h + 2}^n \left( \frac{S_{k \Delta} - S_{(k - 1) \Delta}}{\sqrt{\Delta}} - \left( f(M_{k \Delta}) - f(M_{(k - 1) \Delta}) \right) \sqrt{\Delta} - \sigma Z_{k \Delta} \right)^2, \quad (9).
\]

\subsection{Выбранные данные}

В качестве конкретных примеров я решил взять зависимость цены криптовалюты Ethereum (ETH) от цены криптовалюты Bitcoin (BTC). Данный выбор обосновывается существенной взаимосвязью данных временных рядов, что подтверждается значением корреляции, равным 0.934 на выбранных данных, а также визуальным сходством, представленным на Рис. 6.

\begin{figure}[H]
    \centering
    \includegraphics[width=\textwidth]{/Users/mmenadzhiev/Desktop/Studying/Project/Estimation of the price/Final code/Images/log_btc_eth.png}
    \caption{Динамики логарифмов цен BTC и ETH}
\end{figure}

\subsection{Одношаговое прогнозирование}

Для начала определим \textit{одношаговое} прогнозирование как следующий процесс:

\vspace{-15pt}

\begin{enumerate}
\item Фиксируются \textit{тренировочная} и \textit{тестовая} выборки

\item Модель единожды обучается на \textit{тренировчной} выборке

\item Строится прогноз на \textit{тестовую} выборку

\item Оценивается результативность модели
\end{enumerate}

\vspace{-15pt}

Иными словами, \textit{одношаговое} прогнозирование совпадает с базовым подходом к прогнозированию в машинном обучении.

Также мы будем использовать $\Delta = 1$, поэтому необходимо подобрать только гиперпараметры $h$ и $\sigma$. Для этого воспользуемся \textit{кросс-валидацией} по набору параметров
\[
h \in \{ 5, 10, 20, 50, 100, 200 \}, \quad \sigma \in \{ 0, 0.001, 0.01, 0.03, 0.05 \},
\]
то есть переберем все возможные пары и выберем ту, при которой модель выдаст наилучший результат.







\section{Источники}

[1] StatQuest with Josh Starmer: \href{https://youtu.be/AsNTP8Kwu80?si=N-Snml3hPPGoP0i3}{Recurrent Neural Networks (RNNs)}

[2] StatQuest with Josh Starmer: \href{https://www.youtube.com/watch?v=YCzL96nL7j0&list=PLblh5JKOoLUIxGDQs4LFFD--41Vzf-ME1&index=16&ab_channel=StatQuestwithJoshStarmer}{Long Short-Term Memory (LSTM)}

\end{document}
